\documentclass{article}

\input{~/Documents/school/tex-templates/pset-template.tex}
\lhead{Lucas Brito}
\chead{}
\rhead{One-Dimensional Probability Distribution for Random Walker Position}

\title{One-Dimensional Probability Distribution for Random Walker Position}
\author{Lucas Brito}


\begin{document}
\maketitle

Our model is as follows: say the probability of finding the walker at position
$ x_0 $ at a time $ t_0 +\tau $ $ P(x_0, t_0 +\tau) $. There are three possibilities 
for the walker's position at $ t = t_0 $: the walker was positioned some small 
distance $ \delta $ to the left (then moved to the right), the walker was 
positioned some small distance $ \delta $ to the right (then moved to the left), 
or the walker did not move. With the probability that the walker moves to the 
right being $ p_r $ and the probability the walker moves to the left $ p_\ell $, 
one can set up the equation 
\begin{equation*}
	P(x_0, t_0 + \tau) = p_\ell \cdot P(x_0 + \delta, t_0)
	 + (1-p_r - p_\ell)P(x_0, t_0) + p_r \cdot P(x_0 - \delta, t_0) 
\end{equation*}

We will use a multivariable Taylor expansion about $ x_0, t_0 $. On the left 
hand side, we will naturally evaluate the expansion at $( x_0, t_0+\tau) $; 
all derivatives with respect to $ x $ evaluate to zero for they are in powers 
of $ (x_0 - x_0) $. The left hand side is then 
\begin{align*}
	P(x_0, t_0) + \pdv{P}{t}\evalat_{t=t_0}^{} (t_0 - t_0 + \tau) 
	&+ \frac{1}{2}\pdv[2]{P}{t}\evalat_{t=t_0}^{} (t_0 - t_0 + \tau)^2 + \cdots \\
	&= 
	P(x_0, t_0) + \pdv{P}{t}\evalat_{t=t_0}^{} \tau 
	+ \frac{1}{2}\pdv[2]{P}{t}\evalat_{t=t_0}^{} \tau^2 + \cdots
\end{align*}

The right hand side, evaluated similarly, is 
\begin{align*}
	&p_\ell \qty[ P(x_0, t_0) + \pdv{P}{x}\evalat_{x=x_0}^{} \delta 
		+ \frac{1}{2}\pdv[2]{P}{x} \evalat_{x_0}^{}\delta^2 + \cdots ] \\
	& + (1-p_r - p_\ell)
	\qty[ P(x_0, t_0) + \pdv{P}{x}\evalat_{x=x_0}^{} (x_0-x_0)+
	\pdv[2]{P}{x}\evalat_{x=x_0}^{} (x_0- x_0)^2 + \cdots  ]\\
	& + p_r \qty[ P(x_0, t_0) - \pdv{P}{x}\evalat_{x=x_0}^{} \delta 
		+ \frac{1}{2}\pdv[2]{P}{x} \evalat_{x_0}^{}\delta^2 + \cdots ] 
\end{align*}

Every other term of the right hand side evidently disappears and we are 
left with 
\begin{equation*}
	P(x_0,t_0) + \frac{1}{4}(p_\ell + p_r)
	\pdv[2]{P}{x}\evalat_{x=x_0}^{}\delta^2 + \cdots 
\end{equation*}

With $ \delta $ and $ \tau $ small, we return to the equation and drop 
terms above the second order. 
\begin{equation*}
	P(x_0,t_0) + \pdv{P}{t}\evalat_{t=t_0}^{} \tau + \frac{1}{2}\pdv[2]{P}{t}
	\evalat_{t=t_0}^{} \tau^2 
	= P(x_0, t_0) + \frac{1}{2}(p_\ell + p_r) \pdv[2]{P}{x} \evalat_{x=x_0}^{}\delta^2
\end{equation*}

With $ x_0 $ and $ t_0 $ arbitrary, we are free to drop the evaluation notation: 
\begin{align*}
	P(x_0,t_0) + \pdv{P}{t} \tau + \frac{1}{4}\pdv[2]{P}{t} \tau^2 
	=& P(x_0, t_0) + \frac{1}{2}(p_\ell + p_r) \pdv[2]{P}{x} \delta^2\\
	\pdv{P}{t} \tau + \frac{1}{2}\pdv[2]{P}{t} \tau^2 
	=& \frac{1}{4}(p_\ell + p_r) \pdv[2]{P}{x} \delta^2
\end{align*}

where we've additionally cancelled the constant term. With $ \tau $ and $
\delta $ being arbitrary small quantities, we are free  to set $\tau= \delta^2 $ 
such that 
\begin{equation*}
	\pdv{P}{t} \delta^2 + \frac{1}{2}\pdv[2]{P}{t} \delta^4 
	= \frac{1}{2}(p_\ell + p_r) \pdv[2]{P}{x} \delta^2
\end{equation*}

We previously argued that $ \delta^4 $ terms are too small to be considered, so 
we find 
\begin{equation*}
	\pdv{P}{t} \delta^2 
	= \frac{1}{2}(p_\ell + p_r) \pdv[2]{P}{x} \delta^2
\end{equation*}

and, dividing both sides by $ \delta^2 $, 
\begin{equation*}
	\boxed{
	\pdv{P}{t} 
	= \frac{1}{2}(p_\ell + p_r) \pdv[2]{P}{x} }
\end{equation*}

The one-dimensional heat equation!

\end{document}

